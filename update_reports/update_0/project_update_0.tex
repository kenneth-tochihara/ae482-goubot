% Specify the type of document
\documentclass[a4paper,12pt]{article}

% Load a number of useful packages
\usepackage{graphicx}
\usepackage{amsmath,amssymb,amsfonts,amsthm}
\usepackage{gensymb}
\usepackage[margin=1.0in]{geometry}
\usepackage[colorlinks=true]{hyperref}
\usepackage{cite}
\usepackage[caption=false,font=footnotesize]{subfig}
\usepackage[table]{xcolor}
\usepackage{biblatex}
\usepackage[utf8]{inputenc}
\usepackage{subfig}
\usepackage{textcomp}
\usepackage{amsmath}
\usepackage{float}
\usepackage[labelfont=bf]{caption}
\usepackage{setspace}
\usepackage{siunitx}
\sisetup{output-exponent-marker=\ensuremath{\mathrm{e}}}

\usepackage{titlesec}
\titleformat*{\section}{\large\bfseries}
\titleformat*{\subsection}{\normalsize\bfseries}
\titleformat*{\subsubsection}{\normalsize\bfseries}
\usepackage{indentfirst}

% \usepackage{fancyhdr} 
% \pagestyle{fancy}
% \fancyhf{}
% \fancyheadoffset{0cm}
% \renewcommand{\headrulewidth}{0pt} 
% \renewcommand{\footrulewidth}{0pt}
% \fancyhead[R]{\thepage}


\setlength{\parindent}{0.5in}
\addbibresource{references.bib}


% Two more packages that make it easy to show MATLAB code
\usepackage[T1]{fontenc}
\usepackage{mathptmx}
\usepackage[framed,numbered]{matlab-prettifier}
\lstset{
	style = Matlab-editor,
	basicstyle=\mlttfamily\small,
}

% Say where pictures (if any) will be placed
\graphicspath{{./pictures/}}

% Define title and author (date is auto-generated, unless you define it)
\renewcommand{\baselinestretch}{2}

% Start of document
\begin{document}

\begin{center}
    \Large\textbf{ECE 470: Project Update 0}
    \end{center}
    \newline
    \hspace*{\fill} \normalsize{Team: Jeffery Zhou, Kenneth Tochihara, Charlie Ray} 
    \newline
    \hspace*{\fill} TA: Dhruv Mathur
    \newline
    \hspace*{\fill} Section: Monday 9AM
    \newline
    \hspace*{\fill} Team Name: GouBot
	\pagenumbering{arabic}

\section*{Brief Project Update}\label{introduction}

GouBot is designed to fetch a ball whenever your pet feels lazy and doesn’t want to play. The user will throw the ball and the GouBot will recognize where the ball is using computer vision. The GouBot will plan and execute the path and collect the ball. Finally, the GouBot will return to it’s start location and reorient itself for another throw. The user then can throw another ball for GouBot to fetch, resulting in infinite happiness.

\end{document}